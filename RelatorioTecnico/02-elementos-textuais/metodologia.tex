%
% Documento: Metodologia
%

%\vspace{3cm}%Espaçamento entre linhas	

%\chapter{\textbf{METODOLOGIA}}\label{chap:metodologia}

\section{METODOLOGIA}

\subsection{METODOLOGIA PESQUISA}

Quanto a metodologia de pesquisa, optou-se pela abordagem qualitativa pois, "a pesquisa qualitativa não se preocupa com representatividade numérica, mas, sim, com o aprofundamento da compreensão de um grupo social, de uma organização, etc."  \cite{ufrgs09}, dessa forma pode-se nos aprofundar melhor e entender a realidade dos \emph{stakeholders}. Possui natureza aplicada pois gerará conhecimentos destinados a solução de problemas específicos \cite{ufrgs09} .

Em relação ao procedimento foi adotado o estudo de caso, para \citeonline{yin01} este é uma investigação empírica que investiga um fenômeno contemporâneo dentro de seu contexto da vida real, especialmente quando os limites entre o fenômeno e o contexto não estão claramente definidos, esse foi o procedimento visto que o pesquisador analisou o caso de duas propriedades rurais localizadas no município de Caçapava do Sul e propôs uma solução de software para alguns dos problemas encontrados.

\subsection{METODOLOGIA DE DESENVOLVIMENTO}

Quanto a metodologia de desenvolvimento, utilizou-se a UML (Unified Modeling Language) que, segundo \citeonline{fowler14} é uma família de notações gráficas, apoiadas por um metamodelo único, que ajuda na descrição e no projeto de sistemas de software, particularmente daqueles construídos utlizando o estilo orientado a objetos(OO).

Para tanto utilizou-se do diagrama de casos de uso porque, ele possibilita a compreensão do comportamento externo do sistema, tornando possível ter uma visão das funcionalidades do sistema \cite{guedes18}.

Também foi utilizado o diagrama de atividades, porque através dele é possível "descrever os passos a serem percorridos para a conclusão de uma atividade"  \cite{guedes18}.
