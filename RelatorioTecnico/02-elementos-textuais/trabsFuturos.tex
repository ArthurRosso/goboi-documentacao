%
% Documento: Trabalhos Futuros
%

%\vspace{3cm}%Espaçamento entre linhas	

\chapter{\textbf{TRABALHOS FUTUROS}}\label{chap:trabsFuturos}

Uma análise do processo de criação de bovinos em uma propriedade rural, demonstra que o ciclo de vida do animal necessita de um acompanhamento rigoroso e contínuo. Os registros de informações relativas aos animais adquirem profunda relevância uma vez que a falta de informações pode ocasionar um descontrole sanitário.

Segundo \citeonline{Marcelino16}, na bovinocultura brasileira, seja ela de corte ou de leite, se deve atentar para todos os fatores que possam prejudicar ou diminuir a produção do animal, como por exemplo, as doenças. Muitas  delas podem ser evitadas se os animais forem vacinados, por isso é importante que o produtor esteja sempre atento aos programas de vacinação adotados em cada região, levando em consideração a maneira mais adequada para tratar os animais, pois há vacinas que são aplicadas no rebanho todo, outras são aplicadas somente em certas categorias de animais, selecionando idade e até mesmo o sexo.

A problemática dos pecuaristas, que são o público alvo do presente trabalho, se dá no fato de que embora o registro individual dos animais seja fundamental por conter informações indispensáveis ao manejo do animal, não é essa uma prática habitual por se tratar de uma tarefa muitas vezes complicada, quando feita somente no papel, pois este registro pode ser perdido ou danificado.

% O que precisa ter?
%a) Apresentação do tema e sua delimitação, pequeno histórico do problema, relação com outros estudos;
%b) Justificativa;
%c) Problema;
%d) Objetivos (geral e específicos).
