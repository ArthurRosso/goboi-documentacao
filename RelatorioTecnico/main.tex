\documentclass[oneside]{diretrizes}            % Imprimir apenas frente
%\documentclass[doubleside]{diretrizes}        % Imprimir frente e verso

% Importações de pacotes
\usepackage[alf, abnt-emphasize=bf, recuo=0cm, abnt-etal-cite=2, abnt-etal-list=0]{abntex2cite}  % Citações padrão ABNT
\usepackage[utf8]{inputenc}                         % Acentuação direta
\usepackage[T1]{fontenc}                            % Codificação da fonte em 8 bits
\usepackage{graphicx}                               % Inserir figuras
\usepackage{amsfonts, amssymb, amsmath}             % Fonte e símbolos matemáticos
\usepackage{booktabs}                               % Comandos para tabelas
\usepackage{verbatim}                               % Texto é interpretado como escrito no documento
\usepackage{multirow, array}                        % Múltiplas linhas e colunas em tabelas
\usepackage{indentfirst}                             % Endenta o primeiro parágrafo de cada seção.
\usepackage{microtype}                              % Para melhorias de justificação?
\usepackage[algoruled, portuguese]{algorithm2e}     % Escrever algoritmos
\usepackage{float}                                   % Utilizado para criação de floats
\usepackage{times}                                  % Usa a fonte Times
\linespread{1.5}                                    % Espaçamento entre linhas
\usepackage{graphicx}                               % Adiciona Imagens
\graphicspath{{img/}}                               % Pasta das imagens

% Inclui o preâmbulo do documento
%
% Documento: Preâmbulo
%

\instituicao{Instituto Federal de Educação, Ciência e Tecnologia \\do Rio Grande do Sul}
\abreviatura{IFRS}
\departamento{Campus Canoas}
\local{Canoas}
\programa{Técnico em Informática Integrado ao Ensino Médio}
\nomeautor{Arthur Oliveira de Rosso}
\titulotb{GoBov - Sistema Web de Gerenciamento Bovino}
%\subtitulo{Subtítulo do trabalho}
\data{\today}
\grau{Técnico}
\dataapresentacao{DD/MM/20AA}

%Dados Orientador
\orientador{Rodrigo Perozzo Noll}
\instOrientador{IFRS}
\departamentoorientador{Campus Canoas}
\titulacaoorientador{Prof. Dr.}

%Dados Examinador 1
\nmexamum{Denise Pachman}
\instexamum{IFRS}
\departamentoexamum{Campus Restinga}
\titulacaoexamum{Prof.}

%Dados Examinador 2
\nomeexamdois{Mestre Splinter}
\instexamdois{IFRS}
\departamentoexamdois{Campus Restinga}
\titulacaoexamdois{Prof. Me.}


% Define as cores dos links e informações do PDF
\makeatletter
\hypersetup{
    portuguese,
    colorlinks,
    linkcolor=black,
    citecolor=black,
    filecolor=black,
    urlcolor=black,
    breaklinks=true,
    pdftitle={\@title},
    pdfauthor={\@author},
    pdfsubject={\imprimirpreambulo},
    pdfkeywords={abnt, latex, abntex, abntex2}
}
\makeatother

% Redefinição de labels
\renewcommand{\algorithmautorefname}{Algoritmo}
\def\equationautorefname~#1\null{Equa\c c\~ao~(#1)\null}

% Cria o índice remissivo
\makeindex

% Início do documento
\begin{document}

    % Retira espaço extra obsoleto entre as frases.
    \frenchspacing

    % Elementos pré textuais
    \pretextual
    %
% Documento: Capa
%
\author{Arthur Oliveira de Rosso}
\title{GoBov - Sistema Web de Gerenciamento Bovino.}

\makeatletter
	\begin{center}

		INSTITUTO FEDERAL DE EDUCAÇÃO, CIÊNCIA E TECNOLOGIA

		DO RIO GRANDE DO SUL

		CAMPUS CANOAS

		CURSO TÉCNICO EM INFORMÁTICA INTEGRADO AO ENSINO MÉDIO

		\vfill
		\vfill

		\@author

		\vfill

		\textbf{\@title}

		\vfill

		\textbf{Orientador:} Rodrigo Noll

		\vfill

		Canoas, \today

	\end{center}
\makeatother
               % Capa
    %
% Documento: Resumo (Português)
%

\begin{RESUMO}
\thispagestyle{empty}
	\begin{SingleSpace}

		\hspace{-1.2 cm}  Uma análise do processo de criação de bovinos em uma propriedade rural demonstra que o ciclo de vida do animal necessita de um acompanhamento rigoroso e contínuo. Os registros de informações relativas aos animais adquirem profunda relevância visto que, a falta de informações pode ocasionar um descontrole sanitário. A problemática dos pecuaristas, que são o público alvo do presente trabalho, se dá no fato de que, embora o registro individual dos animais seja fundamental por conter informações indispensáveis ao manejo desses animais, não é essa uma prática habitual, por se tratar de uma tarefa muitas vezes complicada, quando feita somente no papel, este registro pode ser perdido ou danificado. O presente trabalho propõe-se a desenvolver um sistema web, que visa gerenciar animais, proporcionando assim, um controle de medicações, bem como a aplicação de um controle de peso. São objetivos específicos: pesquisar as necessidades dos pecuaristas e de que maneira o sistema pode auxiliá-los, identificar as informações relevantes sobre o ciclo de vida do animal bovino, e por fim, avaliar o resultado do sistema na realidade dos pecuaristas. Durante o levantamento de dados foram buscadas plataformas que trabalham de forma semelhante ao presente sistema, como por exemplo o BovControl, o JetBov e o A3Pecuária. O diferencial do presente trabalho está em ser gratuito e disponibilizar a visualização de relatórios individuais de cada animal. Quanto a metodologia de pesquisa, optou-se pela abordagem qualitativa, pelo fato de ter-se buscado olhar a realidade de fazendeiros de Caçapava do Sul. Possui natureza aplicada pois a plataforma tenta a solução de problemas específicos. O procedimento utilizado foi o estudo de caso, que é uma investigação empírica que investiga um fenômeno dentro de seu contexto, no caso, a realidade dos fazendeiros. Para o desenvolvimento, utilizou-se a UML(Unified Modeling Language) para realizar os diagramas. Optou-se pela UML por se tratar de uma família de notações gráficas, que ajudam na descrição e no projeto de sistemas de software, particularmente daqueles construídos utilizando o estilo orientado a objetos (OO). Como tecnologias e ferramentas foram utilizadas a linguagem de programação Go, para o back-end, juntamente com as bibliotecas Gorilla Mux para as rotas e GORM para o mapeamento objeto-relacional, para o banco de dados foi o MariaDB juntamente com a ferramenta phpMyAdmin para a administração do mesmo, para o front-end foi utilizado HTML (Hypertext Markup Language), CSS (Cascading Style Sheets) e JavaScript com o framework Materialize para estilização das páginas. O sistema encontra-se em fase de finalização para posteriores testes com os pecuaristas que contribuíram no estudo de caso. Como resultados desejados pretende-se que o sistema supra a carência de ferramentas gratuitas e de código aberto para este mercado.

		\vspace*{0.5cm}\hspace{-1.3 cm}\textbf{Palavras-chave}: Bovino. Software. Fazenda. Remédio.



	\end{SingleSpace}
\end{RESUMO}
           % Resumo na língua vernácula
    \sumario

    % Elementos textuais
    \textual
    %
% Documento: Introdução
%

%\vspace{3cm}%Espaçamento entre linhas

\chapter{\textbf{INTRODUÇÃO}}\label{chap:introducao}

Uma análise do processo de criação de bovinos em uma propriedade rural, demonstra que o ciclo de vida do animal necessita de um acompanhamento rigoroso e contínuo. Os registros de informações relativas aos animais adquirem profunda relevância uma vez que a falta de informações pode ocasionar um descontrole sanitário.

Segundo \citeonline{Marcelino16}, na bovinocultura brasileira, seja ela de corte ou de leite, se deve atentar para todos os fatores que possam prejudicar ou diminuir a produção do animal, como por exemplo, as doenças. Muitas  delas podem ser evitadas se os animais forem vacinados, por isso é importante que o produtor esteja sempre atento aos programas de vacinação adotados em cada região, levando em consideração a maneira mais adequada para tratar os animais, pois há vacinas que são aplicadas no rebanho todo, outras são aplicadas somente em certas categorias de animais, selecionando idade e até mesmo o sexo.

A problemática dos pecuaristas, que são o público alvo do presente trabalho, se dá no fato de que embora o registro individual dos animais seja fundamental por conter informações indispensáveis ao manejo do animal, não é essa uma prática habitual por se tratar de uma tarefa muitas vezes complicada, quando feita somente no papel, pois este registro pode ser perdido ou danificado. O registro também pode ser muito complicado pela quantidade de informações.

Desta maneira, o presente sistema propôe o cadastro de animais através de um formulário com um identificador do animal, data de nascimento, peso atual no momento do cadastro, o tipo (pode ser considerado um sinônimo de gênero), raça, finalidade, foto, pai e mãe. Nesses animais é possível um controle de peso, ou seja, em um intervalo de tempo são pesados os animais e observar um gráfico do peso ao longo do tempo. Também é feito um cadastro de remédios, com o nome, validade, descrição e tipo (se trata da via de como será aplicado o remédio). No sistema também é possível registrar as medicações, ou seja, quando os remédios são aplicados nos animais. Através dessas informações o usuário pode manter um controle dos animais e remédios de sua propriedade.
             % Introdução
    %
% Documento: Objetivos
%

%\vspace{3cm}%Espaçamento entre linhas

%\chapter{\textbf{OBJETIVOS}}\label{chap:objetivos}

\section{OBJETIVOS}

O presente capítulo irá apresentar os objetivos deste trabalho. Sendo dividido em objetivo geral e objetivos específicos.

\subsection{OBJETIVO GERAL}

Implementar um sistema web que visa gerenciar os animais de uma propriedade proporcionando um controle sanitário afim de possibilitar a identificação de possíveis focos de doenças e epidemias, bem como a aplicação de um controle de peso capaz de identificar os ganhos obtidos.

\subsection{OBJETIVOS ESPECÍFICOS}

\begin{itemize}
	\item Escolher as tecnologias a serem utilizadas no sistema;
	\item Pesquisar as necessidades dos pecuaristas e de que maneira o sistema pode auxiliá-los;
	\item Modelar o sistema;
	\item Identificar as informações relevantes sobre o ciclo de vida do animal bovino;
	\item Realizar pesquisas de sistemas relacionados para identificar pontos onde há um nicho de mercado;
	\item Implementar o sistema.
	\item Realizar testes do sistema.
	\item Avaliar o sistema na realidade dos pecuaristas.
\end{itemize}
              % Objetivos
    %
% Documento: Metodologia
%

%\vspace{3cm}%Espaçamento entre linhas	

%\chapter{\textbf{METODOLOGIA}}\label{chap:metodologia}

\section{METODOLOGIA}

\subsection{METODOLOGIA PESQUISA}

Quanto a metodologia de pesquisa, optou-se pela abordagem qualitativa pois, "a pesquisa qualitativa não se preocupa com representatividade numérica, mas, sim, com o aprofundamento da compreensão de um grupo social, de uma organização, etc."  \cite{ufrgs09}, dessa forma pode-se nos aprofundar melhor e entender a realidade dos \emph{stakeholders}. Possui natureza aplicada pois gerará conhecimentos destinados a solução de problemas específicos \cite{ufrgs09} .

Em relação ao procedimento foi adotado o estudo de caso, para \citeonline{yin01} este é uma investigação empírica que investiga um fenômeno contemporâneo dentro de seu contexto da vida real, especialmente quando os limites entre o fenômeno e o contexto não estão claramente definidos, esse foi o procedimento visto que o pesquisador analisou o caso de duas propriedades rurais localizadas no município de Caçapava do Sul e propôs uma solução de software para alguns dos problemas encontrados.

\subsection{METODOLOGIA DE DESENVOLVIMENTO}

Quanto a metodologia de desenvolvimento, utilizou-se a UML (Unified Modeling Language) que, segundo \citeonline{fowler14} é uma família de notações gráficas, apoiadas por um metamodelo único, que ajuda na descrição e no projeto de sistemas de software, particularmente daqueles construídos utlizando o estilo orientado a objetos(OO).

Para tanto utilizou-se do diagrama de casos de uso porque, ele possibilita a compreensão do comportamento externo do sistema, tornando possível ter uma visão das funcionalidades do sistema \cite{guedes18}.

Também foi utilizado o diagrama de atividades, porque através dele é possível "descrever os passos a serem percorridos para a conclusão de uma atividade"  \cite{guedes18}.
            % Metodologia
    %
% Documento: Trabalhos relacionados
%

%\vspace{3cm}%Espaçamento entre linhas	

\section{TRABALHOS RELACIONADOS}

% TODO: Nesse capitulo deve aparecer a analise que foi realizada em aplicativos, aplicações, sistemas, plataformas, sites, etc., que guardam semelhança com a proposta a ser desenvolvida no projeto.

Durante o levantamento de dados foram buscadas plataformas que trabalham de forma semelhante ao presente sistema, como por exemplo o BovControl, o JetBov e o A3Pecuária, que serão apresentadas a seguir.

\subsection{BOVCONTROL}

BovControl é uma ferramenta de coleta e análise de dados provenientes da pecuária para melhorar a performance da produção de carne, leite ou genética \cite{bovcontrol10}. 

É disponibilizado em forma de aplicativo e possui um plano gratuito, no qual é possível gerir um rebanho e faz os seguintes manejos nos animais: saída, lote, exame de toque, controle reprodutivo, idade da cria, idade do desmame, medicamento, origem, pesagem, perímetro escrotal, leite, teste diagnóstico, tipo de entrada, vacina, vermifugação. Também é possível visualizar seus dados em um dashboard\footnote{"Dashboards são painéis que mostram métricas e indicadores importantes para alcançar objetivos e metas traçadas de forma visual, facilitando a compreensão das informações geradas." \cite{nascimento17}.}. na internet.

Possui uma parte de relatórios com apenas 2 gráficos, um do total de animais e do tipo de produção do animal (como leite, engorda e genética), e outro do total de animais e do gênero (macho ou fêmea). 

Possui 3 planos profissionais que variam de R\$ 22,99 a R\$ 32,99 por mês. Estes planos incluem gestor financeiro, gestor de tarefas, multiusuários, relatórios personalizados, importação de animais por planilha e estoque de maquinário.  

A Figura 1, é uma imagem mostrando a página inicial do sistema BovControl versão web.

\begin{figure}[H]
	\begin{center}
		\caption{Dashboard do BovControl}
		\includegraphics[width=6in]{../img/bovcontrol.png}

		Fonte: Captura de tela do sistema BovControl.
	\end{center}
\end{figure}


%\begin{figure}[!h]
%\begin{center}
%\caption{BovControl versão mobile - Página inicial}
%\includegraphics[width=6in]{img/bovcontrolapp1.jpg}

% % %\floatfoot{Fonte: Autoria própria.}
%\end{center}
%\end{figure}

%\begin{figure}[!h]
%\begin{center}
%\caption{BovControl versão mobile - Opções de ações}
%\includegraphics[width=6in]{img/bovcontrolapp2.jpeg}

% %\floatfoot{Fonte: Autoria própria.}
%\end{center}
%\end{figure}


\subsection{JETBOV}

Segundo \citeonline{jetbov16}, esse é um sistema que tem como objetivo a gestão da fazenda, da cria até a terminação, a pasto, no semi-confinamento ou confinamento, com um controle de custos com o propósito de aumento da rentabilidade. 

Não possui versão gratuita, porém há uma versão de testes disponível por 21 dias, após isso é necessário realizar um orçamento individual.

O sistema apresenta 2 versões, uma web e outra mobile, a mobile é simples contendo apenas uma página com animais e um botão contendo as opções de manejo como adicionar um novo animal e sua identificação, registros sanitários como vacinações, medicações, exames, vermifugações, etc, adicionar a morte de um animal, o desmame, o parto e pesagem.

A versão web é mais completa contendo um painel de dados da fazenda, com gráficos de animais por sexo, animais por lote, peso por lote e algumas informações como número total de animais da fazenda, peso total da fazenda. 

A Figura 2 apresenta uma imagem mostrando a página inicial do sistema JetBov versão web.

\begin{figure}[H]
	\begin{center}
		\caption{Página inicial da versão web do JetBov}
		\includegraphics[width=6in]{../img/jetbov.png}

		Fonte: Captura de tela do sistema JetBov.
	\end{center}
\end{figure}

%\begin{figure}[!h]
%\begin{center}
%\caption{Jetbov versão mobile - Página inicial}
%\includegraphics[width=6in]{img/jetbovapp1.jpeg}

% %\floatfoot{Fonte: Autoria própria.}
%\end{center}
%\end{figure}

%\begin{figure}[!h]
%\begin{center}
%\caption{JetBov versão mobile - Opções de ações}
%\includegraphics[width=6in]{img/jetbovapp2.jpeg}

% %\floatfoot{Fonte: Autoria própria.}
%\end{center}
%\end{figure}

\subsection{A3PECUÁRIA}

A3Pecuária é um software para gestão de animais com controle de reprodução, pesagens, vacinas e exames, controle financeiro e de estoque e compra e venda. Segundo o site do fabricante: "fornecemos importantes informações de análise de seu rebanho de maneira simples e com uma interface muito fácil de aprender, permitindo gerir seu investimento de forma a aumentar a lucratividade"  \cite{a3pecuaria16}.

Segundo \citeonline{a3pecuaria16}, são 3 tipos de planos, que variam de R\$ 29,90 a R\$ 69,90 por mês, e que gerenciam 500 animais ativos e 1 Fazenda até 3000 animais ativos e fazendas ilimitadas. Não possui versão gratuita, mas uma versão de testes por 30 dias.

Apresenta duas versões, uma web e outra mobile. A mobile, por ser simples, contém apenas a lista de animais da fazenda, o inventário, uma opção de bastão eletrônico e links para a versão web.

A versão web, por ser mais robusta e completa, apresenta uma série de possibilidades de manejos como novo lote, novo animal, nova despesa, nova receita e uma série de análise de dados com relatórios da propriedade.

A seguir, uma imagem mostrando a página inicial do sistema A3Pecuária versão web.


\begin{figure}[!h]
	\begin{center}
		\caption{Página inicial da versão web do A3Pecuária}
		\includegraphics[width=6in]{../img/a3pecuaria.png}

		Fonte: Captura de tela do sistema A3Pecuária.
	\end{center}
\end{figure}

%\begin{figure}[!h]
%\begin{center}
%\caption{A3Pecuária versão mobile - Página inicial}
%\includegraphics[width=6in]{img/a3pecuariaapp1.jpeg}

% %\floatfoot{Fonte: Autoria própria.}
%\end{center}
%\end{figure}

%\begin{figure}[!h]
%\begin{center}
%\caption{A3Pecuária versão mobile - Opções de ações}
%\includegraphics[width=6in]{img/a3pecuariaapp2.jpeg}

% %\floatfoot{Fonte: Autoria própria.}
%\end{center}
%\end{figure}

\subsection{ANÁLISE COMPARATIVA DOS TRABALHOS RELACIONADOS}

A partir da análise feita nas plataformas, pode-se chegar na seguinte conclusão: todas tem seu modo de operação semelhante, como criar, deletar, ler e editar as informações de um animal, e as opções de tratamento de gado (aqui chamado de manejo). As opções de estoque, que no proposto sistema é trabalhado só com medicações e a visualização de relatórios são trabalhados de maneira parecida em todos os sistemas. Dessa maneira o presente sistema tem por objetivo trabalhar com estas mesmas operações, de maneira simples e sem custos.
\begin{table}[]
	\begin{center}
		\caption{Tabela da análise dos trabalhos relacionados}
		\begin{tabular}{ | p{8cm} |  c | c | c | c |}
			\hline
			Funcionalidade & BovControl & JetBov & A3Pecuária & GoBov \\ \hline
			Gerenciamento de animais de uma propriedade & Sim & Sim & Sim & Sim \\  \hline
			Gerenciamento de medicamentos de uma propriedade & Não & Não & Sim & Sim  \\ \hline
			Gerenciamento de medicações de animais & Sim & Sim & Sim & Sim  \\ \hline
			Visualização de relatórios gerais da propriedade & Sim & Sim & Sim & Sim  \\ \hline
			Visualização de relatórios individuais de cada animal & Não & Não & Não & Sim  \\ \hline
			Versão gratuita & Sim & Não & Não & Sim  \\
			\hline
		\end{tabular}
		Fonte: Autoria própria.
	\end{center}
\end{table}      % Trabalhos Relacionados
    %
% Documento: Tecnologias utilizadas
%

\section{TECNOLOGIAS UTILIZADAS}

\begin{itemize}
	\item Golang: É uma linguagem de programação de código aberto que facilita a criação de software simples, confiável e eficiente. <https://golang.org>	
	\item MariaDB: É um banco de dados de código aberto bastante conhecido, é um fork do MySQL. <https://mariadb.org/>
	\item HTML: É uma linguagem de marcação de texto utilizada para desenvolver a estrutura de sites. <https://www.w3.org/html/>
	\item CSS: É uma linguagem de folha de estilos, é util para decorar as páginas HTML. 
<https://www.w3.org/Style/CSS/>
	\item JavaScript: É uma linguagem de programação utilizada que executa scripts no lado do cliente. 
<http://www.ecma-international.org/publications/standards/Stnindex.htm>
	\item Materialize: Framework CSS que facilita as estilizações de páginas HTML. <https://materializecss.com/>
\end{itemize}            % Tecnologias utilizadas
    %
% Documento: Fundamentação Teorica
%

%\vspace{3cm}%Espaçamento entre linhas

\chapter{\textbf{DESCRIÇÃO DA SOLUÇÃO}}\label{chap:descSolucao}

Visando atender o objetivo deste trabalho de desenvolver um sistema capaz de gerenciar animais, uma solução web foi implementada. Este capítulo detalha o desenvolvimento dessa solução, partindo do modelo de requisitos do mesmo.

A análise de requisitos do presente trabalho foi realizada através de conversas com os pecuáristas do estudo de caso. Para delimitar as funcionalidades foi elaborado um diagrama de casos de uso contendo as mesmas.

\section{MODELO DE REQUISITOS}

Para a modelagem de requisitos, utilizou-se o diagrama de casos de uso porque, "ele possibilita a compreensão do comportamento externo do sistema, tornando possível ter uma visão das funcionalidades do sistema" \cite{guedes18}. Neste sentido o diagrama a seguir mostra um usuário desempenhando as funções representadas pelos casos de uso.

\begin{figure}[H]
	\begin{center}
		\caption{Diagrama de Casos de Uso do sistema}
		\includegraphics[width=\textwidth]{../img/casosdeuso.png}

		Fonte: Autoria própria.
	\end{center}
	\label{CdU}
\end{figure}

O caso de uso (CdU) Gerenciar Animal trata das operações realizadas com os animais, no caso, criar um animal, ler as informações dele, atualizar suas informações e deleta-lo. Os casos de uso Gerenciar Remédios e Gerenciar Medicações se referem as mesmas operações, porém, com seus respectivos objetos. O caso de uso Visualizar Relatórios se refere a visualização das informações disponibilizadas pelo sistema, como os gráficos de pesos de animais, aqui chamados de relatórios.


\subsection{\textbf{Telas do Sistema}}\label{telas}
Nesta subseção serão apresentadas as telas do sistema, quais informações elas mostram ao usuário e quais ações elas possibilitam.

\begin{itemize}
\item IV001

A figura 5 é a tela de login do sistema. Nela, o usuário pode se autenticar ou ser direcionado para outra página com o propósito se registrar.
\begin{figure}[H]
	\begin{center}
		\caption{Login no sistema}
		\includegraphics[width=\textwidth]{../img/prototipos/login.png}

		Fonte: Autoria própria.
	\end{center}
\end{figure}

\item IV002

A figura 6 é a tela inicial do sistema. Nela aparecem informações sobre a propriedade, como a quantidade total de animais, remédios e medicações. O usuário também pode ir para a tela de lista de animais, lista de Remédios, ou lista de medicações, ou ser direcionado para o menu, com as preferencias do usuário, nela é possível criar e deletar propósitos, tipos e raças de animais, além dos tipos de remédios.

\begin{figure}[H]
	\begin{center}
		\caption{Página inicial do sistema}
		\includegraphics[width=\textwidth]{../img/prototipos/index.png}

		Fonte: Autoria própria.
	\end{center}
\end{figure}

\item IV003

A figura 7 é a página de lista de animais é apresentada a lista de animais com alguns atributos e 3 ícones, um para deletar o animal, outro para registrar uma medicação e outro para registrar um peso. É mostrada uma barra de consulta, para filtrar os resultados. O usuário pode adicionar um animal, adicionar uma medicação a um animal, pesar um animal, deletar um animal ou ir para a tela de perfil do animal.
\begin{figure}[H]
	\begin{center}
		\caption{Página do animais}
		\includegraphics[width=\textwidth]{../img/prototipos/listaAnimal.png}

		Fonte: Autoria própria.
	\end{center}
\end{figure}

\item IV004

A figura 8 é a página de adicionar animal. O usuário insere as informações do animal e solicita salvar, após isso ele é direcionado para a página de perfil do animal recém adicionado.
\begin{figure}[H]
	\begin{center}
		\caption{Página de adicionar animal}
		\includegraphics[width=\textwidth]{../img/prototipos/addAnimal.png}

		Fonte: Autoria própria.
	\end{center}
\end{figure}


\item IV005

A figura 9 é a página de lista de remédios, é mostrado as informações dos remédios do usuário. Na página o usuário pode adicionar, deletar ou editar um remédio.
\begin{figure}[H]
	\begin{center}
		\caption{Página de remédios}
		\includegraphics[width=\textwidth]{../img/prototipos/listaRemedio.png}

		Fonte: Autoria própria.
	\end{center}
\end{figure}

\item IV006

A figura 10 é a página de adicionar remédio. O usuário insere as informações do remédio e solicita salvar, após isso ele é direcionado para a página da lista de remédios.
\begin{figure}[H]
	\begin{center}
		\caption{Página de adicionar remédio}
		\includegraphics[width=\textwidth]{../img/prototipos/addRemedio.png}

		Fonte: Autoria própria.
	\end{center}
\end{figure}


\item IV007

A figura 11 é a página de lista de medicações, é apresentado as medicações realizadas com uma descrição, a data da realização, a lista de animais medicados e os remédios utilizados. Nela, o usuário pode adicionar ou deletar uma medicação.
\begin{figure}[H]
	\begin{center}
		\caption{Página de medicações}
		\includegraphics[width=\textwidth]{../img/prototipos/listaMedicacao.png}

		Fonte: Autoria própria.
	\end{center}
\end{figure}

\item IV008

A figura 12 é a página de adicionar medicação. Nela o usuário insere uma descrição, a data que a medicação ocorreu, os animais medicados e os remédios utilizados.
\begin{figure}[H]
	\begin{center}
		\caption{Página de adicionar medicação}
		\includegraphics[width=\textwidth]{../img/prototipos/addMedicacao.png}

		Fonte: Autoria própria.
	\end{center}
\end{figure}

\item IV009

A figura 13 é a página de perfil do animal. Nela, são apresentadas todas as informações do animal, inclusive um histórico do que aconteceu com o animal. O usuário pode editar, consultar detalhes, gerenciar uma pesagem do animal ou consultar relatórios individuais do animal.
\begin{figure}[H]
	\begin{center}
		\caption{Página de perfil do animal}
		\includegraphics[width=\textwidth]{../img/prototipos/perfil.png}

		Fonte: Autoria própria.
	\end{center}
\end{figure}


\item IV010

A figura 14 é a página de edição do animal. Nela o usuário pode editar as informações básicas do animal.
\begin{figure}[]
	\begin{center}
		\caption{Página de edição do animal}
		\includegraphics[width=\textwidth]{../img/prototipos/editar.png}

		Fonte: Autoria própria.
	\end{center}
\end{figure}

\newpage
\item IV011

A figura 15 é a página de pesagem. É apresentada a lista de pesagens do animal e o usuário pode gerenciar o peso de um animal.
\begin{figure}[H]
	\begin{center}
		\caption{Página de pesagem do animal}
		\includegraphics[width=\textwidth]{../img/prototipos/addPeso.png}

		Fonte: Autoria própria.
	\end{center}
\end{figure}

\item IV012

A figura 16 é a página de relatórios individuais do animal a qual é apresentado o gráfico do peso do animal ao longo do tempo.
\begin{figure}[H]
	\begin{center}
		\caption{Página de pesagem do animal}
		\includegraphics[width=\textwidth]{../img/prototipos/relatorio.png}

		Fonte: Autoria própria.
	\end{center}
\end{figure}

\end{itemize}

\section{MODELO ENTIDADE RELACIONAMENTO}

O diagrama a seguir mostra o modelo Entidade Relacionamento (ER), que caracteriza a abstração do banco de dados. As tabelas representam a relação do animal com o usuário, com a foto, com o peso, com o propósito, com a raça, com o tipo, e com a remédio, na qual é chamada de medicação. Por sua vez, o remédio também possui uma relação chamada tipo e outra N:N, sendo necessária a criação de uma tabela de ligação, chamada medicação.

\begin{figure}[H]
	\begin{center}
		\caption{Modelo ER}
		\includegraphics[width=\textwidth]{../img/erdoboi.jpg}

		Fonte: Autoria própria.
	\end{center}
	\label{er}
\end{figure}

No próximo capítulo será apresentada a conclusão do presente trabalho com resultados e trabalhos futuros.
            % Descrição da Solução
    %
% Documento: Resultados
%

%\vspace{3cm}%Espaçamento entre linhas

\chapter{\textbf{RESULTADOS E DISCUSSÕES}}\label{chap:resultados}

texto da vergonha, não tem nada aqui
             % Resultados

    % Elementos pós textuais
    \postextual
    \include{./03-elementos-pos-textuais/referencias}        % Referências
    %\printindex                                             % Índice remissivo

\end{document}
