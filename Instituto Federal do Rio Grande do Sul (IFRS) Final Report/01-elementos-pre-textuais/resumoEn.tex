%
% Documento: Resumo (Inglês)
%

\begin{ABSTRACT}
	\begin{SingleSpace}
	
		\hspace{-1.2 cm}  An analysis of the cattle breeding process on a rural property demonstrates that the animal's life cycle requires rigorous and continuous monitoring. Records of information on animals are of profound relevance since lack of information may lead to a lack of sanitary control. The problem of cattle ranchers, who are the target audience of the present study, is that although the individual registration of animals is essential because it contains information essential to the management of these animals, this is not a usual practice, since it is a task often complicated, when done only on paper, since this record can be lost or damaged. The present work proposes to develop a web system, aiming to manage animals, providing a control of medications, as well as the application of a weight control. As for the research methodology, we opted for a qualitative approach, due to the fact that we sought to look at the reality of farmers in Caçapava do Sul. It has an applied nature because the platform tries to solve specific problems. The procedure used was the case study, which is an empirical investigation that investigates a contemporary phenomenon within its real life context, in this case, the reality of the farmers. For the development methodology, the UML was used because it is a family of graphical notations, supported by a unique metamodel, which helps in the description and design of software systems, particularly those built using object oriented style. The desired results are the use and evaluation of the present system by the target public.

		\vspace*{0.5cm}\hspace{-1.3 cm}\textbf{Keywords}: Bovine. Software. Farm. Medicine.
		
		
	\end{SingleSpace}

\end{ABSTRACT}
