%
% Documento: Resumo (Português)
%

\begin{RESUMO}
\thispagestyle{empty}
	\begin{SingleSpace}

		\hspace{-1.2 cm}  O ciclo de vida do animal necessita de um acompanhamento rigoroso e contínuo. Os registros de informações relativas aos animais adquirem profunda relevância visto que, a falta de informações pode ocasionar um descontrole sanitário. A problemática dos pecuaristas, se dá no fato de que, embora o registro individual dos bovinos seja fundamental por conter informações indispensáveis ao manejo dos mesmos, não é essa uma prática usual, por se tratar de uma tarefa  complicada. Quando feita somente no papel, este registro pode ser perdido ou danificado. O presente trabalho tem como objetivo desenvolver um sistema web, que visa gerenciar animais, a fim de proporcionar um controle de medicações, bem como a aplicação do controle de peso. Durante o levantamento de dados foram pesquisadas plataformas que trabalham de forma semelhante ao presente sistema, por exemplo: o BovControl, o JetBov e o A3Pecuária. O presente trabalho se diferencia dos demais por ser gratuito e disponibilizar a visualização dos relatórios individuais de cada bovinos. Quanto a metodologia de pesquisa, foi escolhida a abordagem qualitativa, pelo fato de ter-se buscado analisar a realidade de fazendeiros de Caçapava do Sul. Possui natureza aplicada pois a plataforma tenta a solução dos problemas dos pecuaristas pesquisados. O procedimento utilizado foi o estudo de caso, que se trata de uma investigação empírica de um fenômeno dentro de seu contexto, no caso, a realidade dos fazendeiros. Quanto a metodologia de desenvolvimento, utilizou-se a Unified Modeling Language (UML), para a diagramação, por se tratar de uma família de notações gráficas, que ajudam na descrição e no projeto de sistemas de software. Como tecnologias e ferramentas foram utilizadas a linguagem de programação Go, para o back-end, juntamente com as bibliotecas Gorilla para as rotas e GORM para o mapeamento objeto-relacional. O banco de dados escolhido foi o MariaDB juntamente com a ferramenta phpMyAdmin para a administração do mesmo, para o front-end foi utilizado Hypertext Markup Language (HTML), Cascading Style Sheets (CSS) e JavaScript (JS) com o framework Materialize para estilização das páginas. Foram realizados testes com usuários, e ao final foi aplicado um questionário, no qual foi constatado que a ferramenta supre a carência de ferramentas gratuitas e de código aberto para este mercado.

		\vspace*{0.5cm}\hspace{-1.3 cm}\textbf{Palavras-chave}: Bovino. Software. Fazenda. Remédio.

	\end{SingleSpace}

\end{RESUMO}
