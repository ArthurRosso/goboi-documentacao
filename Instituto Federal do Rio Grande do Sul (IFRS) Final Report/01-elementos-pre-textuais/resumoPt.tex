%
% Documento: Resumo (Português)
%

\begin{RESUMO}
\thispagestyle{empty}
	\begin{SingleSpace}

		\hspace{-1.2 cm}  Uma análise do processo de criação de bovinos em uma propriedade rural demonstra que o ciclo de vida do animal necessita de um acompanhamento rigoroso e contínuo. Os registros de informações relativas aos animais adquirem profunda relevância visto que, a falta de informações pode ocasionar um descontrole sanitário. A problemática dos pecuaristas, que são o público alvo do presente trabalho, se dá no fato de que, embora o registro individual dos animais seja fundamental por conter informações indispensáveis ao manejo desses animais, não é essa uma prática habitual, por se tratar de uma tarefa muitas vezes complicada, quando feita somente no papel, este registro pode ser perdido ou danificado. O presente trabalho propõe-se a desenvolver um sistema web, que visa gerenciar animais, proporcionando assim, um controle de medicações, bem como a aplicação de um controle de peso. São objetivos específicos: pesquisar as necessidades dos pecuaristas e de que maneira o sistema pode auxiliá-los, identificar as informações relevantes sobre o ciclo de vida do animal bovino, e por fim, avaliar o resultado do sistema na realidade dos pecuaristas. Durante o levantamento de dados foram buscadas plataformas que trabalham de forma semelhante ao presente sistema, como por exemplo o BovControl, o JetBov e o A3Pecuária. O diferencial do presente trabalho está em ser gratuito e disponibilizar a visualização de relatórios individuais de cada animal. Quanto a metodologia de pesquisa, optou-se pela abordagem qualitativa, pelo fato de ter-se buscado olhar a realidade de fazendeiros de Caçapava do Sul. Possui natureza aplicada pois a plataforma tenta a solução de problemas específicos. O procedimento utilizado foi o estudo de caso, que é uma investigação empírica que investiga um fenômeno dentro de seu contexto, no caso, a realidade dos fazendeiros. Para o desenvolvimento, utilizou-se a UML(Unified Modeling Language) para realizar os diagramas. Optou-se pela UML por se tratar de uma família de notações gráficas, que ajudam na descrição e no projeto de sistemas de software, particularmente daqueles construídos utilizando o estilo orientado a objetos (OO). Como tecnologias e ferramentas foram utilizadas a linguagem de programação Go, para o back-end, juntamente com as bibliotecas Gorilla Mux para as rotas e GORM para o mapeamento objeto-relacional, para o banco de dados foi o MariaDB juntamente com a ferramenta phpMyAdmin para a administração do mesmo, para o front-end foi utilizado HTML (Hypertext Markup Language), CSS (Cascading Style Sheets) e JavaScript com o framework Materialize para estilização das páginas. O sistema encontra-se em fase de finalização para posteriores testes com os pecuaristas que contribuíram no estudo de caso. Como resultados desejados pretende-se que o sistema supra a carência de ferramentas gratuitas e de código aberto para este mercado.

		\vspace*{0.5cm}\hspace{-1.3 cm}\textbf{Palavras-chave}: Bovino. Software. Fazenda. Remédio.



	\end{SingleSpace}
\end{RESUMO}
