%
% Documento: Resumo (Português)
%

\begin{RESUMO}
\thispagestyle{empty}
	\begin{SingleSpace}
	
		\hspace{-1.2 cm}  Uma análise do processo de criação de bovinos em uma propriedade rural demonstra que o ciclo de vida do animal necessita de um acompanhamento rigoroso e contínuo. Os registros de informações relativas aos animais adquirem profunda relevância visto que, a falta de informações pode ocasionar um descontrole sanitário. A problemática dos pecuaristas, que são o público alvo do presente trabalho, se dá no fato de que embora o registro individual dos animais seja fundamental por conter informações indispensáveis ao manejo desses animais, não é essa uma prática habitual, por se tratar de uma tarefa muitas vezes complicada, quando feita somente no papel, uma vez que este registro pode ser perdido ou danificado. O presente trabalho propõe-se a desenvolver um sistema web, visando gerenciar animais, proporcionando um controle de medicações, bem como a aplicação de um controle de peso. Quanto a metodologia de pesquisa, optou-se pela abordagem qualitativa, pelo fato de ter-se buscado olhar a realidade de fazendeiros de Caçapava do Sul. Possui natureza aplicada pois a plataforma tenta a solução de problemas específicos. O procedimento utilizado foi o estudo de caso, que é uma investigação empírica que investiga um fenômeno contemporâneo dentro de seu contexto da vida real, no caso, a realidade dos fazendeiros. Para a metodologia de desenvolvimento, utilizou-se a UML por se tratar de uma família de notações gráficas, apoiadas por um metamodelo único, que ajuda na descrição e no projeto de sistemas de software, particularmente daqueles construídos utilizando o estilo orientado a objetos. Como resultados desejados tem-se a utilização e avaliação do presente sistema pelo público alvo.

		\vspace*{0.5cm}\hspace{-1.3 cm}\textbf{Palavras-chave}: Bovino. Software. Fazenda. Remédio.
		
		
		
	\end{SingleSpace}
\end{RESUMO}


