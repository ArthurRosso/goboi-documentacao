%
% Documento: Resultados
%

%\vspace{3cm}%Espaçamento entre linhas

\chapter{\textbf{RESULTADOS E DISCUSSÕES}}\label{chap:resultados}

Para que se atenda de forma completa o objetivo geral do trabalho, foram deifinidos objetivos específicos, tais como:
\begin{itemize}
	\item Escolher as tecnologias a serem utilizadas no sistema:
	Foram escolhidas a linguagem de programação Go, o banco de dados MariaDB e HTML, CSS e JS para compor as páginas no lado do cliente. Assim pode-se notar que este objetivo foi concluído com sucesso.

	\item Pesquisar as necessidades dos pecuaristas e de que maneira o sistema pode auxiliá-los com a identificação das informações relevantes sobre o ciclo de vida do animal bovino:
	Na análise dos trabalhos relacionados foi realizada uma comparação para identificar quais são os pontos básicos de um sistema gerenciador bovino e quais pontos os usuários se encontram desamparados, para tanto foram feitas perguntas aos participantes do estudo de caso. Esse objetivo também foi concluído com sucesso.

	\item Modelar o sistema:
	A modelagem pode ser observada na especificação dos casos de uso (Apêndice \ref{especCdU}),  nos diagramas de casos de uso (\ref{CdU}), atividades (Apêndice \ref{ativ}) e no modelo ER (\ref{er}). Por causa disso, o objetivo foi concluído com êxito.


	\item Realizar pesquisas de sistemas relacionados para identificar pontos onde há um nicho de mercado;
	Na análise dos trabalhos relacionados foi identificado que somente o presente sistema é totalmente gratuíto, é o único a possibilitar a visualização de relatórios individuais para cada animal e a exportação para planilhas excel.

	\item Implementar o sistema:


	\item Avaliar o sistema na realidade dos pecuaristas:
	Para atender este objetivo foi feita uma entrevista com um usuário do sistema (Apêndice \ref{chap:entre}), nele foram feitas perguntas quanto a usabilidade e a experiência no sistema, com as respostas foi possível concluir que o usuário conseguiu realizar suas operações de manejo no sistema, e se encontra de acordo com o estado da arte do trabalho.
\end{itemize}
