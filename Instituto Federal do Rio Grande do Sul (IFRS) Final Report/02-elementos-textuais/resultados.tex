%
% Documento: Resultados
%

%\vspace{3cm}%Espaçamento entre linhas

\chapter{\textbf{RESULTADOS E DISCUSSÕES}}\label{chap:resultados}

Para que se atenda de forma completa o objetivo geral do trabalho, foram deifinidos objetivos específicos, tais como:
\begin{itemize}
	\item Escolher as tecnologias a serem utilizadas no sistema:
	\newline
	Foram escolhidas a linguagem de programação Go, o banco de dados MariaDB e HTML, CSS e JS para compor as páginas no lado do cliente. Assim pode-se notar que este objetivo foi concluído com sucesso.

	\item Pesquisar as necessidades dos pecuaristas e de que maneira o sistema pode auxiliá-los com a identificação das informações relevantes sobre o ciclo de vida do animal bovino:
	\newline
	Na análise dos trabalhos relacionados foi realizada uma comparação para identificar quais são os pontos básicos de um sistema gerenciador bovino e quais pontos os usuários se encontram desamparados, para tanto foram feitas perguntas aos participantes do estudo de caso. Esse objetivo também foi concluído com sucesso.

	\item Modelar o sistema:
	\newline
	A modelagem pode ser observada na especificação dos casos de uso (Apêndice \ref{ap:especCdU}),  nos diagramas de casos de uso (\ref{CdU}), atividades (Apêndice \ref{ap:ativ}) e no modelo ER (\ref{er}). Por causa disso, o objetivo foi concluído com êxito.


	\item Realizar pesquisas de sistemas relacionados para identificar pontos onde há um nicho de mercado:
	\newline
	Na análise dos trabalhos relacionados foi identificado que somente o presente sistema é totalmente gratuíto, é o único a possibilitar a visualização de relatórios individuais para cada animal e a exportação para planilhas excel.

	\item Implementar o sistema:
	\newline
	Seguiu-se o cronograma e através disso foi possível completar esse objetivo.

	\item Avaliar o sistema na realidade dos pecuaristas:
	\newline
	Para atender este objetivo foi feita uma entrevista com um usuário do sistema (Apêndice \ref{ap:entre}), nele foram feitas perguntas quanto a usabilidade e a experiência no sistema, com as respostas foi possível constatar que o usuário considerou as funcionalidades do trabalho úteis e conseguiu realizar suas operações de manejo no sistema, o que corrobora para com o cumprimento dos objetivos do trabalho.
\end{itemize}

\section{TRABALHOS FUTUROS}\label{chap:trabsfuturos}

O trabalho foi apresentado em eventos como VI IFCITEC, a Feira de Ciência e Inovação Tecnológica, realizado no Campus Canoas do Instituto Federal de Educação, Ciência e Tecnologia (IFRS) no dia 19 de setembro de 2018 e a ENPEX, Salão de Ensino, Pesquisa e Extensão do IFRS. Nesses eventos foram constatados algumas funcionalidades que ficaram como trabalhos futuros como a possibilidade de inserir no sistema o tipo de nutrição do animal e fazer um aplicativo para que se possa utilizar o sistema sem necessitar o acesso a internet visto que nem todas as propriedades tem acesso a mesma.
