%
% Documento: Fundamentação Teorica
%

%\vspace{3cm}%Espaçamento entre linhas	

\chapter{\textbf{FUNDAMENTAÇÃO TEÓRICA}}\label{chap:fundTeorica}

As tecnologias presentes hoje no mundo digital possibilitam um aceso muito mais ágil
e eficiente à informação do que a 20 anos atrás, por exemplo. Ao analisar o processo de
interação entre o autor de um artigo na Web e um leitor comum, é possível interpretar que
o leitor é o sujeito que absorve os conhecimentos do autor, contudo, ele não somente
absorve, mas recria. “A interatividade direta e indireta entre autores e leitores na internet
torna possível a participação do leitor na reconstrução de um conteúdo através da adoção
da filosofia do compartilhamento social com respeito à questão de autoria. ” (ZANAGA,
2008).