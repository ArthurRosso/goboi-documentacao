%
% Documento: Metodologia
%

%\vspace{3cm}%Espaçamento entre linhas

%\chapter{\textbf{METODOLOGIA}}\label{chap:metodologia}

\section{METODOLOGIA}\label{chap:metodologia}

Esta seção irá mostrar as metodologias do presente trabalho. Sendo divididas em metodologia de pesquisa, com abordagem qualitativa e natureza aplicada com procedimento estudo de caso.

\subsection{\textbf{Metodologia de Pesquisa}}

Quanto a metodologia de pesquisa, optou-se pela abordagem qualitativa pois, "a pesquisa qualitativa não se preocupa com representatividade numérica, mas, sim, com o aprofundamento da compreensão de um grupo social, de uma organização, etc."  \cite{ufrgs09}, dessa forma pode-se nos aprofundar melhor e entender a realidade do grupo pesquisado. Possui natureza aplicada pois gerará conhecimentos destinados a solução de problemas específicos \cite{ufrgs09} .

Em relação ao procedimento foi adotado o estudo de caso, para \citeonline{yin01} este é uma investigação empírica que investiga um fenômeno contemporâneo dentro de seu contexto da vida real, especialmente quando os limites entre o fenômeno e o contexto não estão claramente definidos.

Tal procedimento foi  escolhido uma vez que o pesquisador analisou o caso de duas propriedades rurais localizadas no município de Caçapava do Sul no interior do Rio Grande do Sul. A partir de pesquisas  feitas com fazendeiros locais, foram identificados alguns problemas como: a dificuldade de gerenciar informações da vida completa dos animais sem perda de  dados e falta de mobilidade dos mesmos, visto que, informações escritas a mão podem ser perdidas facilmente. Logo, foi pensado um software, a fim de resolver tal problema.

Ao longo do desenvolvimento do trabalho também foram feitas entrevistas e conversas com os mesmos pecuáristas. Nessas conversas foram constatados alguns problemas que foram sendo corrigidos no decorrer do desenvolvimento do software.

\subsection{\textbf{Desenvolvimento}}

Quanto ao desenvolvimento, utilizou-se a linguagem UML que, segundo \citeonline{fowler14} é uma linguagem apoiada por um modelo único, que ajuda na descrição e no projeto de sistemas de software, particularmente daqueles construídos utlizando o paradigma orientado a objetos.

Foi utilizado o diagrama de atividades, porque através dele é possível "descrever os passos a serem percorridos para a conclusão de uma atividade"  \cite{guedes18}. As especificações de casos de uso funcionam como ponte entre o diagrama de casos de uso e atividades transformando as atividades realizadas no sistema em texto, mais descritivo e sucinto.
