%
% Documento: Introdução
%

%\vspace{3cm}%Espaçamento entre linhas

\chapter{\textbf{INTRODUÇÃO}}\label{chap:introducao}

Uma análise do processo de criação de bovinos em uma propriedade rural, demonstra que o ciclo de vida do animal necessita de um acompanhamento rigoroso e contínuo. Os registros de informações relativas aos animais adquirem profunda relevância uma vez que a falta de informações pode ocasionar um descontrole sanitário.

Segundo \citeonline{Marcelino16}, na bovinocultura brasileira, seja ela de corte ou de leite, se deve atentar para todos os fatores que possam prejudicar ou diminuir a produção do animal, como por exemplo, as doenças. Muitas  delas podem ser evitadas se os animais forem vacinados, por isso é importante que o produtor esteja sempre atento aos programas de vacinação adotados em cada região, levando em consideração a maneira mais adequada para tratar os animais, pois há vacinas que são aplicadas no rebanho todo, outras são aplicadas somente em certas categorias de animais, selecionando idade e até mesmo o sexo.

A problemática dos pecuaristas, que são o público alvo do presente trabalho, se dá no fato de que embora o registro individual dos animais seja fundamental por conter informações indispensáveis ao manejo do animal, não é essa uma prática habitual por se tratar de uma tarefa muitas vezes complicada, quando feita somente no papel, pois este registro pode ser perdido ou danificado. O registro também pode ser muito complicado pela quantidade de informações.

Desta maneira o presente sistema propôe o cadastro de animais através de um formulário com um identificador do animal, data de nascimento, peso atual no momento do cadastro, o tipo (pode ser considerado um sinônimo de gênero), raça, finalidade, foto, pai e mãe. Nesses animais é possível um controle de peso, ou seja, em um intervalo de tempo são pesados os animais e é possível observar um gráfico do peso ao longo do tempo. Também é feito um cadastro de remédios, através de um formulário também, no sistema com o nome, validade, descrição e tipo (se trata da via de como será aplicado o remédio). No sistema também é possível registrar as medicações, ou seja, quando é aplicado os remédios nos animais. Através dessas informações o usuário pode manter um controle dos animais e remédios de sua propriedade.
